\documentclass[10pt,a4paper]{article}
\usepackage{sltc2016}
\usepackage{url}
\usepackage{framed}
% use microtype if available
\IfFileExists{microtype.sty}{\usepackage{microtype}}{}
\usepackage{color}
\usepackage{fancyvrb}
\newcommand{\VerbBar}{|}
\newcommand{\VERB}{\Verb[commandchars=\\\{\}]}
\DefineVerbatimEnvironment{Highlighting}{Verbatim}{commandchars=\\\{\}
,fontsize=\small}
% Add ',fontsize=\small' for more characters per line
\definecolor{shadecolor}{RGB}{248,248,248}
\newenvironment{Shaded}{\begin{snugshade}}{\end{snugshade}}
\newcommand{\KeywordTok}[1]{\textcolor[rgb]{0.13,0.29,0.53}{\textbf{{#1}}}}
\newcommand{\DataTypeTok}[1]{\textcolor[rgb]{0.13,0.29,0.53}{{#1}}}
\newcommand{\DecValTok}[1]{\textcolor[rgb]{0.00,0.00,0.81}{{#1}}}
\newcommand{\BaseNTok}[1]{\textcolor[rgb]{0.00,0.00,0.81}{{#1}}}
\newcommand{\FloatTok}[1]{\textcolor[rgb]{0.00,0.00,0.81}{{#1}}}
\newcommand{\CharTok}[1]{\textcolor[rgb]{0.31,0.60,0.02}{{#1}}}
\newcommand{\StringTok}[1]{\textcolor[rgb]{0.31,0.60,0.02}{{#1}}}
\newcommand{\CommentTok}[1]{\textcolor[rgb]{0.56,0.35,0.01}{\textit{{#1}}}}
\newcommand{\OtherTok}[1]{\textcolor[rgb]{0.56,0.35,0.01}{{#1}}}
\newcommand{\AlertTok}[1]{\textcolor[rgb]{0.94,0.16,0.16}{{#1}}}
\newcommand{\FunctionTok}[1]{\textcolor[rgb]{0.00,0.00,0.00}{{#1}}}
\newcommand{\RegionMarkerTok}[1]{{#1}}
\newcommand{\ErrorTok}[1]{\textbf{{#1}}}
\newcommand{\NormalTok}[1]{{#1}}

\begin{document}

\date{\today}
\title{Outsourcing morphology in Grammatical Framework: a case study for Hungarian}
\name{Inari Listenmaa}
\address{Department of Computer Science and Engineering \\ University of Gothenburg and Chalmers University of Technology \\ \tt inari@chalmers.se}


\abstract{We implement a miniature resource grammar in Grammatical Framework (GF) by using 
resources developed in the Apertium community: a finite-state morphological 
transducer and a disambiguation grammar. 
Our goals are twofold: to share resources within the rule-based community, as 
well as to prevent the GF grammar growing in size. Especially for languages with 
complex morphology, not having to store large inflection tables makes the grammar
smaller and faster. As for development effort, we hope that the external resources
would also save time in the grammar writing process. The next steps are to scale up
to a full resource grammar, and parametrise the grammar for different tagsets.}

\maketitleabstract


\section{Introduction}
\label{sec:introduction}
Grammatical Framework (GF, \cite{ranta2004jfp}) is a grammar formalism and a programming 
language for writing multilingual grammars.
A GF grammar consists of an \emph{abstract syntax}, which declares the
categories and constructions in the grammar, and a number of \emph{concrete
  syntaxes}, where the categories and constructions are implemented,
separately for each language.
By allowing multiple concrete syntaxes for a single abstract syntax, GF is a natural 
choice for interlingual translation.

One of the most important contributions of GF is its Resource Grammar Library 
\cite{ranta2009lilt}, which contains 31 languages as of September 2016. All the languages 
share the same core abstract syntax, and each language can have an extra module for 
constructions that are particular to that language. Via the common core, we can 
translate basic syntactic structures between any pair of the 31 languages; any new 
language added to the library will be connected to all of the existing languages.

Typically, writing a morphological description is the first step in starting a resource grammar.
A new grammar writer can easily spend weeks or months in defining inflection classes 
and writing morphophonological transformation rules for their language. 
Such a description in GF may well be valuable on its own right: it may provide 
insights about the morphological complexity of a language
\cite{detrez_ranta2012eacl}, or even lend itself for language description.
However, creating a morphological description in GF is time-consuming, and the result 
is often less efficient than a finite-state description.
If a new RGL language already has morphological resources, using them would ideally
speed up both development and performance of the resource grammar.
%\footnote{Of course, we are not denying anyone their fun of writing a morphological description in GF.}


\section{Implementation}

We implemented a miniature version (44 functions) of the GF resource
grammar \cite{ranta2009lilt} for this experiment. 
%At this stage, the grammar is not suited for more than toy
%applications, but the effort of grammar writing seems promising.
The development of this initial grammar took just a
couple of hours; the author has no knowledge of Hungarian, but years
of experience with GF and Apertium.

In the following sections, we present briefly the traditional way of writing GF
grammars, followed by the new method.

\subsection{Traditional GF grammar}

A self-contained GF grammar stores inflection tables in each lexical
entry. Then, the syntactic functions select the appropriate forms from
the tables. We will illustrate the process with a grammar that has
four categories: NP, V, VP and S, and two syntactic functions: complementation
and predication. The \texttt{Compl} function combines a NP and a V
into a VP, and \texttt{Pred} combines a NP and a VP into an S.
The abstract syntax of this grammar is shown below.
 
\vspace{-1mm}

\begin{Shaded}
\begin{Highlighting}[]
  \NormalTok{abstract }\DataTypeTok{Grammar} \FunctionTok{=} \NormalTok{\{}

  \NormalTok{cat }
    \DataTypeTok{S} \NormalTok{; }\DataTypeTok{NP} \NormalTok{; }\DataTypeTok{VP} \NormalTok{; }\DataTypeTok{V} \NormalTok{;}

  \NormalTok{fun}
    \DataTypeTok{Pred} \FunctionTok{:} \DataTypeTok{NP} \OtherTok{->} \DataTypeTok{VP} \OtherTok{->} \DataTypeTok{S} \NormalTok{;}
    \DataTypeTok{Compl} \FunctionTok{:} \DataTypeTok{V} \OtherTok{->} \DataTypeTok{NP} \OtherTok{->} \DataTypeTok{VP} \NormalTok{;}
  \NormalTok{\}}
\end{Highlighting}
\end{Shaded}

% \begin{verbatim}
%   abstract Grammar = {

%   cat 
%     S ; NP ; VP ; V ;

%   fun
%     Pred : NP -> VP -> S ;
%     Compl : V -> NP -> VP ;
%   }
% \end{verbatim}

\vspace{-1mm}

Next, we write a concrete syntax for a language where verbs inflect
for agreement and nouns for case. 
We introduce the parameters \texttt{Case} and \texttt{Agr}, and create
the categories specific to this language.

\vspace{-1mm}

\begin{Shaded}
\begin{Highlighting}[]
  \NormalTok{param}
    \DataTypeTok{Case} \FunctionTok{=} \DataTypeTok{Nom} \FunctionTok{|} \DataTypeTok{Acc} \FunctionTok{|} \DataTypeTok{Dat} \FunctionTok{|} \FunctionTok{...} \NormalTok{;}
    \DataTypeTok{Agr} \FunctionTok{=} \DataTypeTok{SgP1} \FunctionTok{|} \DataTypeTok{SgP2} \FunctionTok{|} \FunctionTok{...} \FunctionTok{|} \DataTypeTok{PlP3} \NormalTok{;}
  \NormalTok{lincat }
    \DataTypeTok{S}  \FunctionTok{=} \DataTypeTok{Str} \NormalTok{;}
    \DataTypeTok{VP} \FunctionTok{=} \DataTypeTok{Agr} \OtherTok{=>} \DataTypeTok{Str} \NormalTok{;}
    \DataTypeTok{V} \FunctionTok{=} \NormalTok{\{ s }\FunctionTok{:} \DataTypeTok{Agr} \OtherTok{=>} \DataTypeTok{Str} \NormalTok{; compl }\FunctionTok{:} \DataTypeTok{Case} \NormalTok{\} ;}
    \DataTypeTok{NP} \FunctionTok{=} \NormalTok{\{ s }\FunctionTok{:} \DataTypeTok{Case} \OtherTok{=>} \DataTypeTok{Str} \NormalTok{; agr }\FunctionTok{:} \DataTypeTok{Agr} \NormalTok{\} ;}
\end{Highlighting}
\end{Shaded}

%\begin{verbatim}
%  param
%    Case = Nom | Acc | Dat | ... ;
%    Agr = SgP1 | SgP2 | ... | PlP3 ;
%  lincat 
%    S  = Str ;
%    VP = Agr => Str ;
%    V = { s : Agr => Str ; compl : Case } ;
%    NP = { s : Case => Str ; agr : Agr } ;
%\end{verbatim}

Then we implement the functions which operate on the given categories.
The complementation function must choose a right case from the object
NP, depending on the verb. Likewise, the predication function must
choose the right agreement from the VP, depending on the subject NP.

\vspace{-1mm}

\begin{Shaded}
\begin{Highlighting}[]
  \NormalTok{lin }
    \DataTypeTok{Compl} \NormalTok{v obj }\FunctionTok{=}
     \NormalTok{table \{ agr }\OtherTok{=>} \NormalTok{verb}\FunctionTok{.}\NormalTok{s }\FunctionTok{!} \NormalTok{agr}
                 \FunctionTok{++} \NormalTok{obj}\FunctionTok{.}\NormalTok{s }\FunctionTok{!} \NormalTok{v}\FunctionTok{.}\NormalTok{compl \} ;}

    \DataTypeTok{Pred} \NormalTok{subj vp }\FunctionTok{=} \NormalTok{subj }\FunctionTok{!} \DataTypeTok{Nom} 
                \FunctionTok{++} \NormalTok{vp }\FunctionTok{!} \NormalTok{subj}\FunctionTok{.}\NormalTok{agr ;}
\end{Highlighting}
\end{Shaded}

% \begin{verbatim}
% lin 
%   Compl verb obj =
%    table { agr => verb.s ! agr
%                ++ obj.s ! verb.compl } ;

%   Pred subj vp = subj ! Nom 
%               ++ vp ! subj.agr ;

% \end{verbatim}

\noindent These parameters prevent the grammar from overgenerating.
Given the NP \emph{a fa} `the tree' and the verb
\emph{szeret} `love', the grammar will only accept \emph{a fa szereti a
  f\'{a}t} `the tree loves the tree', no other combinations of cases and verb inflections.

On the high level, this grammar design looks simple. 
But under the hood, the GF source code is compiled into a low-level
format. Adding a new value to a parameter or a new field to a record 
does not change much in the high-level rules, but each addition
multiplies the number of low-level rules. As a result, GF grammars for
morphologically complex languages often suffer from performance issues.

% Nouns, verbs and other inflecting categories contain large tables, for all
%  combinations of number and case. The verb, likewise, contains all the
%  conjugation, including tense, mood, person, aspect, and even
%  non-finite forms (e.g. {\em singing}). 

\subsection{GF grammar with external resources}
\label{sec:GFext}

In the version with an external morphological analyser, all lexical
entries contain only a base form, along with a POS tag. The features,
which were used to select right forms from the inflection tables, are
now replaced with string fields for \emph{tags}, such as following.

\begin{Shaded}
\begin{Highlighting}[]
  \NormalTok{lincat}
    \DataTypeTok{S}  \FunctionTok{=} \DataTypeTok{Str} \NormalTok{;}
    \DataTypeTok{VP} \FunctionTok{=} \NormalTok{\{ verb }\FunctionTok{:} \DataTypeTok{Str} \NormalTok{; obj }\FunctionTok{:} \DataTypeTok{Str} \NormalTok{\} ;}
    \DataTypeTok{NP} \FunctionTok{=} \NormalTok{\{ s }\FunctionTok{:} \DataTypeTok{Str} \NormalTok{; agrTag }\FunctionTok{:} \DataTypeTok{Str} \NormalTok{\} ;}
    \DataTypeTok{V2} \FunctionTok{=} \NormalTok{\{ s }\FunctionTok{:} \DataTypeTok{Str} \NormalTok{; complTag }\FunctionTok{:} \DataTypeTok{Str} \NormalTok{\} ;}
\end{Highlighting}
\end{Shaded}


Then, the syntactic functions can be reduced into concatenating tags,
and taking care of the right word order. Below are the same syntactic
functions for complementation and predication.


\begin{Shaded}
\begin{Highlighting}[]
  \NormalTok{lin }
    \DataTypeTok{Compl} \NormalTok{v o }\FunctionTok{=} \NormalTok{\{ verb }\FunctionTok{=} \NormalTok{v}\FunctionTok{.}\NormalTok{s ;}
                  \NormalTok{obj }\FunctionTok{=} \NormalTok{o}\FunctionTok{.}\NormalTok{s }\FunctionTok{+} \NormalTok{v}\FunctionTok{.}\NormalTok{complTag \} ;}

    \DataTypeTok{Pred} \NormalTok{subj vp }\FunctionTok{=} \NormalTok{(subj }\FunctionTok{+} \StringTok{"<nom>"}\NormalTok{)}
                \FunctionTok{++} \NormalTok{(vp}\FunctionTok{.}\NormalTok{verb }\FunctionTok{+} \NormalTok{subj}\FunctionTok{.}\NormalTok{agrTag)}
                \FunctionTok{++} \NormalTok{vp}\FunctionTok{.}\NormalTok{obj ;}
\end{Highlighting}
\end{Shaded}

% The entries in the GF grammar consist of just base lemmas with tags
% added: for example, the accusative and dative forms of the second
% person singular pronoun are stored as \texttt{te<prn><p2><sg><acc>}
% and \texttt{te<prn><p2><sg><dat>} instead of their inflected forms,
% \emph{téged} and \emph{neked} respectively.
With this setup, the GF trees are linearised into what looks like
Apertium analyses: for example,
\texttt{mi<prn><pers><p1><mf><pl><nom>} \texttt{ j\'{a}r<vblex><past><p3><pl>} for `we walked'.

\subsection{Pipeline}

The user of the grammar types in normal Hungarian words, and the input is analysed 
by the external morphological analyser, which is further disambiguated by a 
Constraint Grammar \cite{karlsson1995constraint}. Only then is the sentence given to the GF grammar, 
which will return the syntactic parse tree.
The grammar can also be used in translation from another language
to Hungarian. First the source language is analysed into a GF tree,
then the tree is linearised into Hungarian with tags. To get actual
Hungarian, the morphological analyser is called to generate
the inflected forms from the tags.



\section{Discussion}

Adding external tools into a GF grammar  
makes the system more complex. In comparison, a 100 \% GF grammar is
easier to understand, and immune to eventual changes in the other systems. 
%Hence a grammar
%writer may decide to value independence and clarity over the convenience of reusing resources.
%Furthermore, we have only implemented a fragment of the resource grammar:
%scaling up to the full grammar may reveal challenges in the approach,
%which would increase the development effort.
If we only wanted to speed up development, we could use the FS morphology
inside GF: generate all forms of the FS lexicon, and convert it into GF
inflection tables. This solution would save the time in writing
morphological rules, but not affect the size of the grammar.
% \footnote{Aarne Ranta (personal
%   communication) notes that the original reason to develop
%   morphology in GF was that FS morphologies were proprietary.}
%This frees us from writing morphological operations in GF; however, it would not solve the
%problem of performance. 
If there is already a GF morphology, we can also convert it into a FS
morphology, via LEXC output from GF: this would aid performance, but not development.

There is another potential benefit of a GF grammar with tags: mapping from one tagset to another.
%One of the most unnecessary annoyances in computational linguistics is
%tagset incompatibility. 
To demonstrate, Table~\ref{table:engem} shows two analyses of the Hungarian first
person singular pronoun \emph{engem} `me', in Morphdb.hu \cite{tron2006morphdb} and Apertium.
If we want to support both tagsets, we can write a
second concrete syntax; or better yet, use the GF module system to
parametrise over the concrete tags and the functions to combine
them. The latter will require more work than the straight-forward
solution shown in Section~\ref{sec:GFext}, but it is preferable to duplicating the work in syntax.

\begin{table}[t]
\center
    \begin{tabular}{r|l}
   Morphdb.hu  & \texttt{\'{e}n/NOUN<PERS<1>><CAS<ACC>>} \\ \hline
   Apertium    & \texttt{\'{e}n<prn><pers><p1><mf><sg><acc>} \\
    \end{tabular}
\label{table:engem}
\caption{Analyses for the pronoun \emph{engem} `me'}
\end{table}

\vspace{-2mm}

\section{Conclusion}

The experiment has been small, but promising in terms of effort.
We estimate it would take days to write similar fragment in full GF,
especially for a developer who doesn't know the language; in contrast,
this grammar took merely hours to write.
Evaluating the performance and correctness, as well as scaling up to a
full resource grammar, is left for future work.
In conclusion, we are happy with the results, and hope the experiment
will encourage more sharing of resources within the rule-based community.



\bibliographystyle{sltc2016}
\bibliography{GF-RG.bib}


\end{document}
